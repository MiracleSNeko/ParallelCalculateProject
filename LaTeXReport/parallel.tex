\documentclass{article}

\usepackage[UTF8]{ctex}
\usepackage{amsmath}
\usepackage{amsthm}
\usepackage{bm}
\usepackage{extarrows}
\usepackage{color}
\usepackage{mathrsfs}
\usepackage{amssymb}
\usepackage{booktabs}
\usepackage{algorithm} 
%\usepackage{algorithmic} 
\usepackage{algorithmicx}  
\usepackage{algpseudocode}  
\usepackage{listings}
\usepackage{multirow}
\usepackage{graphicx}
\usepackage{xcolor}
\usepackage[width=16cm]{geometry}


\geometry{a4paper,scale=0.8}
\newtheorem{theorem}{\hspace{2em}定理}
\newtheorem{lemma}{\hspace{2em}引理}
%\newtheorem{proof}{证明}
\floatname{algorithm}{算法}  
\renewcommand{\algorithmicrequire}{\textbf{输入:}}  
\renewcommand{\algorithmicensure}{\textbf{输出:}}  


\title{并行算法试题 答题卷}
\author{林宇健   2018202296}
\date{\today}

\begin{document}
\maketitle

%%%%%%%%%%%%%%%%%%%%%%%%%%%%%%%%%%%%%%%%%%%%%%%%%%%%%%%%%%%%%%%%%%%%%%%%%
%%%%%%%%%%%%%%%%%%%%%%%%%%%%%%%%%%%%%%%%%%%%%%%%%%%%%%%%%%%%%%%%%%%%%%%%%
\section{并行矩阵向量乘法}
\subsection{问题描述和分析}
编程计算$Ax$,其中$A$是$m\times n$的稠密矩阵,$x$是$n$维列向量,分别采用$1,4,8,16$台处理机计算。给出并行算法,及并行效率分析。

记$y=Ax$。为简便起见,取$A$是$n\times n$的方阵,$n = 2048$可以整除$1,4,8,16$,从而保证每个进程储存的向量块维度相同。对于不能整除$4,8,16$的$n$,可通过循环存储等方式为各个进程分配矩阵和向量的数据。在计算时编程随机生成了$matrix_{2048\times 2048}$和$vector_{2048 \times 1}$作为待计算的矩阵和向量(计算程序略去)。

我们采用一维行划分的方式并行计算矩阵向量乘法。假设矩阵$A$按逐行一维块划分为$p$个块($p$表示进程数),即$A = [A_{1}, A_{2}, \cdots, A_{p}]^{T},A_{k} = [A_{k,0},A_{k,1},\cdots,A_{k,p}]$。其中

\begin{center}	
	$A_{k,j} = $
	\begin{bmatrix}
		$a_{k\times n/p+1,j\times n/p+1}$ & $a_{k\times n/p+1,j\times n/p+2}$ & $\cdots$ & $a_{k\times n/p+1,j\times n/p+n/p}$ \\
		$a_{k\times n/p+2,j\times n/p+1}$ & $a_{k\times n/p+2,j\times n/p+2}$ & $\cdots$ & $a_{k\times n/p+2,j\times n/p+n/p}$ \\
		$\vdots$ & $\vdots$ & $\ddots$ & $\vdots$ \\
		$a_{k\times n/p+n/p,j\times n/p+1}$ & $a_{k\times n/p+n/p,j\times n/p+2}$ & $\cdots$ & $a_{k\times n/p+n/p,j\times n/p+n/p}$ \\
	\end{bmatrix}
\end{center}

与之相对应,向量$x$和向量$y$也分为$p$个块,其第$k(0\leq k\leq p-1)$个块分别为

$$$$
$$$$

\begin{center}
	\renewcommand\tabcolsep{7pt}
	\begin{tabular}{cccc|c|c|ccc}
		\multicolumn{4}{c}{\bf{$A$}}& \multicolumn{1}{c}{ }& \multicolumn{1}{c}{\bf{$X$}}& &\\
		\cline{1-4} \cline{6-6}
		\multicolumn{1}{|c}{\bf{$A_{1,1}$}}& $A_{1,2}$& $\cdots$& $A_{1,p}$& & $x_{1}$& &$P_{1}$\\
		\cline{1-4}\cline{6-6}
		\cmidrule[0.25pt]{1-8}
		\cline{1-4}\cline{6-6}
		\multicolumn{1}{|c}{\bf{$A_{2,1}$}}& $A_{2,2}$& $\cdots$& $A_{2,p}$& & $x_{2}$& &$P_{2}$\\
		\cline{1-4}\cline{6-6}
		\cmidrule[0.25pt]{1-8}
		\cline{1-4}\cline{6-6}
		\multicolumn{1}{|c}{$\vdots$}& $\vdots$& $\ddots$& $\vdots$& & $\vdots$& &$\vdots$\\
		\cline{1-4}\cline{6-6}
		\cmidrule[0.25pt]{1-8}
		\cline{1-4}\cline{6-6}
		\multicolumn{1}{|c}{\bf{$A_{p,1}$}}& $A_{p,2}$& $\cdots$& $A_{p,p}$& & $x_{p}$& &$P_{p}$\\
		\cline{1-4}\cline{6-6}
	\end{tabular}
\end{center}
	
\begin{center}
	\setlength{\fboxsep}{0.3pt}
	\lstset{
		language={[90]Fortran},
		extendedchars=false,
		lineskip=-1pt,
		basicstyle=\ttfamily\footnotesize\color{black},
		keywordstyle=\bf,
		showstringspaces=true,
		numberstyle=\color{black},
		%texcl=false,
		escapeinside=``,
		numbers=left,
		frame=single
	}
	\lstinputlisting{matrix_mul_vector_parallel.f90}
	\lstinputlisting{matrix_mul_vector_serial.f90}
\end{center}	

\begin{center}
	\lstset{
		language={C},
		extendedchars=false,
		lineskip=-1pt,
		basicstyle=\ttfamily\footnotesize\color{black},
		keywordstyle=\color{black},
		showstringspaces=true,
		numberstyle=\color{black},
		%texcl=false,
		escapeinside=``,
		numbers=left,
		frame=single
	}
	\lstinputlisting{autoexec.sh}
\end{center}	

\begin{center}
	\lstset{
		language={[77]Fortran},
		extendedchars=false,
		lineskip=-1pt,
		basicstyle=\ttfamily\footnotesize\color{black},
		keywordstyle=\scriptsize\bf,
		showstringspaces=false,
		numberstyle=\color{black},
		%texcl=false,
		escapeinside=``,
		numbers=left,
		frame=single
	}
	\lstinputlisting{walltime.for}
\end{center}	



\end{document}